\subsection{Partially centralized educators}
\label{apx:pools}

In the Section~\ref{sec:priv-chain} we have concluded that the cost of the
private block proof publication should depend on the size of the corresponding
Merkle tree. This is done in order to scale spendings of different educators
with amount of data they produce and store in their private blockchains.

But this solution has a disadvantage: the Witnesses are more incentivized to
include proofs from large educators in public blocks rather than from small
educators, as proofs from large educators contain more fees. If block size is
limited, it may lead to delays of inclusion of small educators' proofs in the
public blockchain.

In order to resolve this problem, small educators can use trusted third-party
services (e. g. \url{teachmeplease.com}) for interacting with Disciplina
platform instead of running Educator nodes by themselves. But this means that
third-party service has access to all the educator's data, including marks and
assignments of her students, and also receives all the revenue for trading this
data. Some small educators might find this option unacceptable.

Therefore, we propose a mechanism of \textit{educator pools}, which allow small
educators to delegate block proof publishing to a third party in a partially trustless
way.

\begin{figure}[ht]
  \centering
  \includegraphics[width=\textwidth]{pool-hierarchical-mtree}
  \caption{Two-level hierarchical sized Merkle tree for a block published by a pool
    manager}
  \label{fig:pools:lvl2block}
\end{figure}

The idea is the following:

\begin{itemize}
\item Every small Educator still maintains her own small private chain
\item When a small Educator forms a block in her private chain, she sends the
  block header to a third party called \textit{pool manager} instead of
  publishing it directly to Witnesses. Another difference is that Educator
  should also send a separate signature for her \textit{ATG delta} (if it's not
  empty).
\item A \textit{pool manager} collects block headers from Educators until
  total number of transactions in all Educator's blocks is more than some
  threshold $K_{min}$.
\item Then pool manager builds a sized Merkle tree over the list of received
  Educator's block headers, forming a \textit{second-level block}
  (Fig.~\ref{fig:pools:lvl2block}). The header of second-level block gets published
  on the public blockchain. Instead of containing a single \textit{ATG delta},
  the header of this second-level block contains a list of separate signed
  \textit{ATG deltas} of small educators. We assume that this approach would
  not create a problem of oversized block headers because Educators don't
  typically create and close courses very often, and an average number of ATG
  deltas in every single block header will stay small.
\item After constructing a second-level block, pool manager sends each of the
  small Educators a path to their block headers in a second-level sized Merkle tree.
\item Having this path, each Educator can construct a publicly verifiable proof
  for any transaction in her private block by simply concatenating this path
  with a path to transaction in a first-level Merkle tree.
\end{itemize}

For every processed block header small educator pays pool manager a fee
calculated by this formula:
\begin{equation}\label{apx:pools:pools-cost-eq}
  C_{pool}(B) = \alpha_{pool} + \beta_{pub} \cdot N_{tr}(B)
\end{equation}
where $\beta_{pub}$ is a network price coefficient from
\ref{sec:priv-chain:pub-cost-eq}, and $\alpha_{pool}$ is a constant fee set by
the pool manager.

If a pool manager sets such $\alpha_{pool}$ that
$\alpha_{pool} < \alpha_{pub}$, but $\alpha_{pool} N > \alpha_{pub}$, then for
every published second-level block a pool manager gains
$\alpha_{pool} N - \alpha_{pub}$ coins, while every Educator in pool pays less
for the block header publishing then if published directly to Witnesses.
Therefore, every participant has an incentive to remain in the pool.

However, such approach poses a threat of excessive centralization in the network
-- it might happen so that \textit{all} educators would find it beneficial to
join one huge pool, which will be able to provide the lowest pool fees due to
amount of its clients. In order to avoid this, we might change the formula
\ref{sec:priv-chain:pub-cost-eq} in a way which diminishes the profits of
joining the pool for big educators:

\begin{equation}\label{apx:pools:poosl-cost-eq-2}
  C_{pub}(B) = \alpha_{pub} \cdot \mathbbm{1}_{\{N_{tr} < K_{tr}\}} + \beta_{pub}
  \cdot N_{tr}
\end{equation}
where $K_{tr}$ is a sufficiently big threshold value. This formula ensures that
educators which publish large blocks don't pay constant $\alpha_{pub}$ coins for
every block, which makes it unprofitable for them to use pools.

Note that a small Educator participating in the pool should slightly change the
structure of his own chain. Every block header should contain a hash of previous
one (as described in Section~\ref{sec:priv-chain}), but blocks of an Educator
participating in the pool are published only as part of second-level pooled
block. So the educator's block header should contain a header hash of a
second-level block containing previous first-level block in educator's chain
instead of header hash of previous first-level block.

We don't yet provide a fully trustless solution for pooling. Small Educators
should trust a pool manager to provide correct second-level proofs of their
blocks, and in theory an adversarial pool manager may cheat on Educators by not
including their block headers in a published second-level block after she
received the money. However, we assume that deceiving the small educators and
losing them as a long-term clients is far less profitable than being honest.
Nevertheless, we are currently working on a fully trustless pooling protocol
based on a smart contract which ensures that a pool manager cannot deceive any
Educator.

The idea of pooling also improves the performance of the system, because it
reduces the amount of information published on public chain. Lots of private
block headers issued by lots of educators are replaced by few private block
headers issued by large educators and pools.
