\section{Introduction}
Recent advances in blockchain technology and decentralized consensus systems open up new possibilities for building untamperable domain-specific ledgers with no central authority. Since the launch of Bitcoin \cite{nakamoto2008bitcoin} blockchains had been primarily used as a mechanism for value transfers. With the growth of the Ethereum platform \cite{wood2014ethereum}, the community realized that by using a chain of blocks and consensus rules one can not only store value and track its movement, but, more generally, store some state and enforce conditions upon which this state can be modified.

Bitcoin, Ethereum and other permissionless blockchains were developed with the assumption that everyone is free to join the network and validate transactions, that are public. However, the industry often requires privacy, and thus the permissive solutions with private ledgers came to exist. These solutions include Tendermint \cite{kwon2014tendermint}, Hyperledger \cite{cachin2016architecture}, Kadena \cite{Kadena} and others.

The increased interest and the variety of the blockchain technologies lead to the growth of their application domains. The idea of storing educational records in the blockchain has been circulating in the press and academic papers for several years. For example, \cite{swan2015blockchain} and \cite{devine2015blockchain} focus on the online education and propose to create a system based on the educational smart contracts in a public ledger. Recently, Sony announced a project that aims at incorporating educational records in a permissioned blockchain based on Hyperledger \cite{SGE}. The ledger is going to be shared between major offline educational institutes.

The main issue these solutions have in common is that they target a certain subset of ways people get knowledge. We propose a more general approach that would unite the records of large universities, small institutes, schools and online educational platforms to form a publicly verifiable chain. Contrary to the solutions like Ethereum, we do not aim at proposing a programmable blockchain that fits all the possible applications. Rather, we believe, that we should harness all the latest knowledge that emerged in the last few years in the fields of consensus protocols, authenticated data structures and distributed computations to offer a new domain-specific ledger. In this paper we introduce Disciplina — the platform based on blockchain technology that aims to transform the way educational records are generated, stored and accessed.
